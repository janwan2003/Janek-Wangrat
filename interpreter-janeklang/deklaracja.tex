\documentclass[12pt]{article}
\usepackage{geometry}
\geometry{a4paper, margin=1in}
\usepackage[utf8]{inputenc}
\usepackage[T1]{fontenc}
\usepackage[polish]{babel}
\usepackage{polski}
\usepackage{booktabs}

\title{Deklaracja języka --- Janeklang}
\author{Jan Wangrat}
\date{\today}

\begin{document}

\maketitle

\section*{Kilka słów}
Moim językiem będzie zmodyfikowany Latte, tak więc składniowo jest to pewna mieszanka Javy i C++.
Wprowadzam kilka marginalnych zmian, które są nieuniknione w celu spełnienia warunków 
w tabelce cech, z tego samego powodu usuwam rzeczy, które uważam za niepotrzebne (w celu
spełnienia własności z tabelki). Moim celem jest uzyskanie 27 punktów. 

\section*{Dodane funkcjonalności}

\begin{itemize}
    \item \textbf{Definicje zmiennych na poziomie TopDef:} Możliwość definiowania zmiennych globalnych z inicjalizacją wartości na najwyższym poziomie struktury programu.
    \item \textbf{Zagnieżdżone definicje funkcji:} Możliwość definiowania funkcji wewnątrz innych funkcji lub bloków kodu.
    \item \textbf{Break i Continue} Dodanie do statementów breaka i continue
\end{itemize}

\section*{Usunięte funkcjonalności}

\begin{itemize}
  \item \textbf{Inkrementacja i Dekrementacja:} Usunięto operatory `++` i `--`.
  \item \textbf{Usunięcie deklraracji:} Usunięcie delaracji, każda zmienna musi zostać zaiinicjalizowana.
\end{itemize}


\begin{table}[h]
    \centering
    \begin{tabular}{@{}ll@{}}
    \toprule
    Funkcja                                                          & Zrealizowano \\ \midrule
    01 (trzy typy)                                                   & +            \\
    02 (literały, arytmetyka, porównania)                            & +            \\
    03 (zmienne, przypisanie)                                        & +            \\
    04 (print)                                                       & +            \\
    05 (while, if)                                                   & +            \\
    06 (funkcje lub procedury, rekurencja)                           & +            \\
    07 (przez zmienną / przez wartość / in/out)                      & +            \\
    08 (zmienne read-only i pętla for)                               &              \\
    09 (przesłanianie i statyczne wiązanie)                          & +            \\
    10 (obsługa błędów wykonania)                                    & +            \\
    11 (funkcje zwracające wartość)                                  & +            \\
    12 (4) (statyczne typowanie)                                     & +            \\
    13 (2) (funkcje zagnieżdżone ze statycznym wiązaniem)            & +            \\
    14 (1/2) (rekordy/listy/tablice/tablice wielowymiarowe)          &              \\
    15 (2) (krotki z przypisaniem)                                   &              \\
    16 (1) (break, continue)                                         & +            \\
    17 (4) (funkcje wyższego rzędu, anonimowe, domknięcia)           &              \\
    18 (3) (generatory)                                              &              \\ \bottomrule
    \end{tabular}
    \caption{Tabelka Cech, razem: 27 punktów}
\end{table}
    

  
\end{document}
